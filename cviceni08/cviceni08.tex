\documentclass{beamer}
\usetheme[pageofpages=z,
          bullet=circle,
          titleline=true,
          alternativetitlepage=true,
          titlepagelogo=uplogo,
          ]{UVT}

\usepackage[utf8x]{inputenc}
\usepackage{czech}
\usepackage{minted}
% \usepackage{url}


\newenvironment{itemizex}%
  {\large \begin{itemize}%
    \setlength{\itemsep}{8pt}%
    \setlength{\parskip}{8pt}}%
  {\end{itemize}}

\newenvironment{enumeratex}%
  {\large \begin{enumerate}%
    \setlength{\itemsep}{6pt}%
    \setlength{\parskip}{6pt}}%
  {\end{enumerate}}

\newenvironment{itemizey}%
  {\large \begin{itemize}%
    \setlength{\itemsep}{6pt}%
    \setlength{\parskip}{6pt}}%
  {\end{itemize}}


\title{Základy programování 2 (KMI/ZP2 a KMI/UP2)}
\subtitle{Cvičení 8: Práce s textovými soubory}
\author{Lukáš Havrlant}
\date{2. dubna 2013}
\institute{Univerzita Palackého}

\begin{document}

\begin{frame}[t,plain]
\titlepage
\end{frame}


\begin{frame}[t,fragile]\frametitle{Otevření souboru} 
    \begin{itemizex}
        \item Všechny potřebné funkce jsou v knihovně \texttt{stdio.h}
        \item K otevření souboru slouží funkce \texttt{fopen}:
\begin{minted}[linenos=true]{c} 
FILE *fopen(const char* path, const char* mode);
\end{minted}
        \item Funkce vrací ukazatel na strukturu \texttt{FILE} -- ukazatel na \uv{datový proud}, anglicky \uv{stream}.
        \item Pokud otevření selže, funkce vrací \texttt{NULL}.
        \item Parametr \texttt{path} je cesta k souboru.
    \end{itemizex}
\end{frame}


\begin{frame}[t,fragile]\frametitle{Mód} 
Musíme specifikovat, co chceme se souborem dělat: jestli chceme číst, přepisovat nebo zapisovat na konec souboru. 

\begin{tabular}{cl}
\texttt{"r"} & otevře soubor pro čtení\\
\texttt{"w"} & vytvoří nový soubor pro zápis (pokud soubor existuje, smaže ho)\\
\texttt{"a"} & otevře soubor pro zápis na konec souboru\\
\end{tabular}

Dále vždy existují varianty s pluskem: \texttt{r+}, \texttt{w+}, \texttt{a+}, které otevírají soubor jak pro čtení, tak pro zápis. 

\begin{minted}[linenos=true]{c} 
fopen("seznam.txt", "r"); // otevre soubor pro cteni
// pokud soubor existuje, smaze jeho obsah
// a otevre ho jak pro cteni, tak pro zapis
fopen("seznam.txt", "w+"); 
// otevre soubor pro zapis, nesmaze obsah
// a zapisovat bude na konec souboru
fopen("seznam.txt", "a"); 
\end{minted}
\end{frame}


\begin{frame}[t,fragile]\frametitle{Textový vs. binární režim} 
    \begin{itemizex}
        \item Ke všem předchozím módům ještě můžeme přidat \texttt{t} nebo \texttt{b}. Mód \texttt{t} značí textový mód, mód \texttt{b} je binární. 
        \item Výchozí je textový.
        \item Příklady:
\begin{minted}[linenos=true]{c} 
// otevre soubor pro cteni v textovem modu
fopen("seznam.txt", "rt");

// otevre soubor pro cteni a zapis v binarnim modu
fopen("seznam.dat", "r+b");
\end{minted}
    \end{itemizex}
\end{frame}


\begin{frame}[t,fragile]\frametitle{Čtení z textového souboru} 
    \begin{itemizex}
        \item Ke čtení jednoho znaku slouží funkce \texttt{getc}:
\begin{minted}[linenos=true]{c} 
int getc(FILE* soubor);
\end{minted}
        \item Předpokládejme, že \texttt{soubor.txt} obsahuje:
\begin{minted}[linenos=true]{bash} 
Prvni radek,
druhy radek.
\end{minted}

\begin{minted}[linenos=true]{c} 
FILE* soubor = fopen("soubor.txt", "r");
printf("%c\n", getc(soubor)); // P
printf("%c\n", getc(soubor)); // r
printf("%c\n", getc(soubor)); // v
\end{minted}
    \end{itemizex}
\end{frame}


\begin{frame}[t,fragile]\frametitle{Čtení jednoho řádku} 
    \begin{itemizex}
        \item Slouží k tomu funkce \texttt{fgets}
\begin{minted}[linenos=true]{c} 
char* fgets(char* radek, int max, FILE* soubor);
\end{minted}
        \item Funkce \texttt{fgets} načte ze souboru jeden řádek (včetně \texttt{'\textbackslash n'} znaku) a tyto znaky uloží do parametru \texttt{radek}. 
        \item Parametr \texttt{radek} už musí mít alokovanou paměť.
        \item \texttt{fgets} načte maximálně \texttt{max - 1} znaků, je-li řádek delší, nenačte ho celý. (-1 proto, že na konci musí ještě uložit \texttt{'\textbackslash0'})
    \end{itemizex}
\end{frame}


\begin{frame}[t,fragile]\frametitle{Příklad čtení ze souboru} 
\begin{minted}[linenos=true]{c} 
FILE* soubor = fopen("soubor.txt", "r");
char* radek = (char*) malloc(50 * sizeof(char));
fgets(radek, 50, soubor);
printf("%s\n", radek); // Prvni radek,(+ novy radek)
fgets(radek, 5, soubor);
printf("%s\n", radek); // druh
\end{minted}


\begin{itemizex}
    \item Všimněte si, že druhý \texttt{fgets} nezačal číst od začátku souboru, ale pokračoval tam, kde první \texttt{fgets} skončil. 
    \item Druhý \texttt{fgets} také přečetl pouze 4 znaky, 5. použil na uložení ukončovací nuly. 
\end{itemizex}
\end{frame}



\begin{frame}[t,fragile]\frametitle{Formátované čtení} 
    \begin{itemizex}
        \item Funkce \texttt{fscanf}, obdoba funkce \texttt{scanf}, kterou známe. 
\begin{minted}[linenos=true]{c} 
int fscanf(FILE *soubor, const char* format, ...);
\end{minted}
        \item Pro soubor obsahující řádek \texttt{10 3.14!} bychom dostali:
\begin{minted}[linenos=true]{c} 
int x;
double des;
char ch;
FILE* soubor = fopen("soubor.txt", "r");
fscanf(soubor, "%i %lf%c", &x, &des, &ch);
printf("%i, %g, %c\n", x, des, ch); // 10, 3.14, !
\end{minted}
    \end{itemizex}
\end{frame}


\begin{frame}[t,fragile]\frametitle{Jak poznat konec souboru} 
    \begin{itemizex}
        \item Pokud jsme již narazili na konec souboru, vrací funkce \texttt{getc} a \texttt{fscanf} konstantu \texttt{EOF} (\uv{End of file}).
        \item Funkce \texttt{fgets} vrací \texttt{NULL}.
\begin{minted}[linenos=true]{c} 
char znak;
FILE* soubor = fopen("soubor.txt", "r");
while((znak = getc(soubor)) != EOF) {
    printf("%c\n", znak);
}
// program vypise kazdy znak souboru 
// na samostatny radek a skonci
\end{minted}
    \end{itemizex}
\end{frame}


\begin{frame}[t,fragile]\frametitle{Navrácení znaku do proudu} 
    \begin{itemizey}
        \item Častým požadavkem je: \uv{čti znaky, dokud nenarazíš na číslici}. Což ale znamená, že se zarazíme až ve chvíli, kdy nějakou číslici přečteme.
        \item Pokud čteme soubor s obsahem \texttt{abcdef123456}, zůstane nám ve zbytku proudu obsah \texttt{23456}.
        \item Funkce \texttt{ungetc} vrací znak do proudu. 
\begin{minted}[linenos=true]{c} 
int ungetc(int znak, FILE* soubor);
// pouziti:
if (je_cislice(nacteny_znak)) {
    ungetc(nacteny_znak, soubor);
}
\end{minted}
    \end{itemizey}
\end{frame}


\begin{frame}[t,fragile]\frametitle{Zápis do textového souboru} 
    \begin{itemizex}
        \item Používáme obdobné funkce jako při čtení.
        \item Funkce \texttt{puts} slouží k zápisu jednoho znaku.
\begin{minted}[linenos=true]{c} 
int putc(int znak, FILE* soubor);
\end{minted}

\begin{minted}[linenos=true]{c} 
FILE* soubor = fopen("soubor.txt", "w");
putc('4', soubor);
putc('2', soubor);
\end{minted}
        \item Tímto přepíšeme obsah souboru \texttt{soubor.txt} na obsah \texttt{"42"}.
    \end{itemizex}
\end{frame}


\begin{frame}[t,fragile]\frametitle{Zápis do textového souboru v "a" módu} 
    \begin{itemizex}
        \item Zapisovat můžeme i za konec souboru:
\begin{minted}[linenos=true]{c} 
FILE* soubor = fopen("soubor.txt", "a");
putc('4', soubor);
putc('2', soubor);
\end{minted}
        \item Obsah souboru by se změnil takto:
\begin{minted}[linenos=true]{bash} 
Prvni radek,
druhy radek.42
\end{minted}
    \end{itemizex}
\end{frame}


\begin{frame}[t,fragile]\frametitle{Zápis textového řetězce} 
    \begin{itemizex}
        \item Funkce \texttt{fputs}:
\begin{minted}[linenos=true]{c} 
int fputs(const char* retezec, FILE* soubor);
\end{minted}
        \item Zapíše do souboru celý řetězec vyjma ukončovací nuly.
\begin{minted}[linenos=true]{c} 
FILE* soubor = fopen("soubor.txt", "w");
fputs("ahoj", soubor);
fputs(" svete", soubor);
// Obsah souboru bude:
// ahoj svete
\end{minted}
    \end{itemizex}
\end{frame}


\begin{frame}[t,fragile]\frametitle{Formátovaný zápis} 
    \begin{itemizex}
        \item Funkce \texttt{fprintf}:
\begin{minted}[linenos=true]{c} 
int fprintf(FILE* soubor, const char* format, ...);
\end{minted}
        \item V podstatě stejné jako funkce \texttt{printf}, akorát se zapisuje do souboru.
\begin{minted}[linenos=true]{c} 
FILE* soubor = fopen("soubor.txt", "w");
fprintf(soubor, "%i -- %g", 12, 3.0 / 7);
// Obsah souboru bude:
// 12 -- 0.428
\end{minted}
    \end{itemizex}
\end{frame}


\begin{frame}[t,fragile]\frametitle{Uzavření souboru} 
    \begin{itemize}
        \item \textbf{Po ukončení práce se souborem vždy musíme soubor uzavřít pomocí funkce \texttt{fclose}:} 
\begin{minted}[linenos=true]{c} 
int fclose(FILE* soubor);
\end{minted}
        \item Proč uzavírat? 
            \begin{itemize}
                \item Aby se skutečně zapsal obsah na disk. Data se po použití zapisovacích funkcí ukládají do mezipaměti, zápis na disk je drahá záležitost, proto se to OS snaží optimalizovat.
                \item Mohou existovat limity na počet otevřených souborů pro jeden program.
                \item Jiný program může chtít zapisovat do stejného souboru.
            \end{itemize}
    \end{itemize}
\begin{minted}[linenos=true]{c} 
FILE* soubor = fopen("soubor.txt", "w");
// zapis, cteni, cokoliv... 
fclose(soubor);
\end{minted}
\end{frame}


\begin{frame}[t,fragile]\frametitle{Vynucení zápisu bez uzavření souboru} 
    \begin{itemizex}
        \item Pokud chceme vynutit uložení dat z mezipaměti na disk, můžeme použít funkci \texttt{fflush}:
\begin{minted}[linenos=true]{c} 
int fflush (FILE* soubor);
\end{minted}

\begin{minted}[linenos=true]{c} 
FILE* soubor = fopen("soubor.txt", "w");
fprintf(soubor, "%i -- %g", 12, 3.0 / 7);
fflush(soubor);
fputs("dalsi zapis", soubor);
fclose(soubor);
\end{minted}
        \item \texttt{fflush} ve skutečnosti nevynucuje zápis na disk, vynucuje jen uvolnění mezipaměti. Data ale mohou skončit v další mezipaměti. 
    \end{itemizex}
\end{frame}


\begin{frame}[t,fragile]\frametitle{Odchytávání chyb} 
    \begin{itemizex}
        \item Tohle je noční můra a zlý sen dohromady. 
        \item Jakákoliv práce s diskem může selhat. Soubor nemusí existovat, program nemusí mít práva daný soubor otevřít, disk může být plný\dots
        \item Po každém volání některé z funkcí bychom tak měli otestovat, zda vše proběhlo v pořádku. 
    \end{itemizex}
\end{frame}


\begin{frame}[t,fragile]\frametitle{Pokud dojde k chybě} 
    \begin{itemizex}
        \item Funkce \texttt{putc}, \texttt{fputs}, \texttt{getc}, \texttt{fscanf}, \texttt{fclose} a \texttt{ungetc} vrací při chybě \texttt{EOF}.
        \item Funkce \texttt{fgets}, \texttt{fopen} vrací \texttt{NULL}.
        \item Takže kód by měl vypadat nějak takto: (na dalším slajdu)
    \end{itemizex}
\end{frame}


\begin{frame}[t,fragile]\frametitle{Kód s ošetřením chyb} 
\begin{minted}[linenos=true]{c} 
FILE* soubor = fopen("soubor.txt", "r");
if (soubor != NULL) {
    if (putc('!', soubor) == EOF) {
        // osetreni chyby
    }
    if (fclose(soubor) == EOF) {
        // osetreni chyby
    }
} else {
    // osetreni chyby
}
\end{minted}
\end{frame}


\begin{frame}[t,fragile]\frametitle{Typ chyby} 
    \begin{itemizex}
        \item \dots a to ještě v předchozím kódu nijak nezjišťujeme, jaká chyba vlastně nastala. 
        \item V knihovně \texttt{errno.h} existuje globální proměnná \texttt{errno}, ve které je uloženo číslo chyby. 
        \item Funkce \texttt{perror} dokáže vypsat chybovou hlášku. 
\begin{minted}[linenos=true]{c} 
void perror(const char* s);
\end{minted}
        \item Řetězec, který bere jako parametr, vypíše před chybovou hláškou. (příklad na dalším slajdu)
    \end{itemizex}
\end{frame}


\begin{frame}[t,fragile]\frametitle{Příklad s výpisem typu chyby} 
\begin{minted}[linenos=true]{c} 
FILE* soubor = fopen("neexistujici_soubor", "r");
if (soubor != NULL) {
    // udelej neco se souborem...
} else {
    perror("Chyba");
    printf("ID chyby: %i\n", errno);
    switch (errno) {
        case // a jedeme
    }
}
// Vystupem bude:
// Chyba: No such file or directory
// ID chyby: 2
\end{minted}
\end{frame}


\begin{frame}[t,fragile]{Úlohy}
\begin{center}
\vskip 1cm
{\Large Seznam úloh je na \url{http://KMIzp2.jdem.cz/}}
\vskip 2cm
\url{http://tux.inf.upol.cz/~havrlant/}
\end{center}
\end{frame}


\end{document}