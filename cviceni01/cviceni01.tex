\documentclass{beamer}
\usetheme[pageofpages=z,
          bullet=circle,
          titleline=true,
          alternativetitlepage=true,
          titlepagelogo=uplogo,
          ]{UVT}

\usepackage[utf8x]{inputenc}
\usepackage{czech}
\usepackage{minted}
% \usepackage{url}


\newenvironment{itemizex}%
  {\large \begin{itemize}%
    \setlength{\itemsep}{8pt}%
    \setlength{\parskip}{8pt}}%
  {\end{itemize}}

\newenvironment{enumeratex}%
  {\large \begin{enumerate}%
    \setlength{\itemsep}{6pt}%
    \setlength{\parskip}{6pt}}%
  {\end{enumerate}}


\title{Základy programování 2 (KMI/ZP2 a KMI/UP2)}
\subtitle{Cvičení 1: Rekurzivní funkce}
\author{Lukáš Havrlant}
\date{12. února 2013}
\institute{Univerzita Palackého}

\begin{document}

\begin{frame}[t,plain]
\titlepage
\end{frame}


\begin{frame}[t,fragile]{Konzultace}
\begin{itemizex}
\item V pracovně 5.076.
\item Každé úterý 14.45--16.15. 
\item Emaily:
  \begin{itemizex}
    \item lukas.havrlant+skola@gmail.com
    \item lukas.havrlant@upol.cz
  \end{itemizex}
\item Web: \url{http://tux.inf.upol.cz/~havrlant/}
\end{itemizex}
\end{frame}


\begin{frame}[t,fragile]{Doporučená literatura}
\begin{itemize}
\item Pavel Herout: Učebnice Jazyka C. Kopp, 2007.
\item Pavel Herout: Učebnice Jazyka C, 2. díl. Kopp.
\item Kernighan, Ritchie: Programovací jazyk C, 2006.
\item \dots dále viz Portál
\item + strýček Google
\end{itemize}
\end{frame}

\begin{frame}[t,fragile]{Podmínky získání zápočtu}
\begin{itemizex}
  \item Vypracovat alespoň 8 domácích úkolů, které budou zadávány na každém cvičení. 
  \item Vypracovat závěrečný projekt (= jeden větší domácí úkol). 
  \item Docházka není povinná.
\end{itemizex}
\end{frame}

\begin{frame}[t,fragile]\frametitle{Funkce: opakování} 
\begin{minted}[linenos=true]{c} 
void upozorni() { // Definice funkce
    printf("Pockej, zapomnela jsi na okurku!");
}

double prumer(double a, double b) {
    return (a + b) / 2;
}
prumer(2.5, 7.5); // Volani funkce, vysledek: 5.0

double absolutni_hodnota(double x) {
    if (x >= 0)  return x;
    else         return -x;
}
\end{minted}
\end{frame}



\begin{frame}[t,fragile]{Rekurzivní funkce}
\begin{itemizex}
    \item Funkce, která ve svém těle volá sama sebe. 
    \item Slouží k podobným účelům jako cykly.
    \item Obsahuje limitní podmínku (jako cykly). 
    \item Cyklus bývá většinou rychlejší. Rekurze může být přehlednější. 
    \item Teorii znáte z Paradigmat programování. 
  \end{itemizex}
\end{frame}


\begin{frame}[t,fragile]\frametitle{Faktoriál} 
Faktoriál se zapisuje $n!$ a definuje se, pro $n\in\mathbb{N}_0$, jako
$$n!=1\cdot2\cdot3\cdot\ldots\cdot n,\quad \,\mbox{neboli}\quad 
n!=
\begin{cases}
n\cdot(n-1)!&\mbox{pokud } n > 0\\
1&\mbox{pokud } n = 0
\end{cases}
$$
Řešení v C:
\begin{minted}[linenos=true]{c} 
int factorial(int n)
{
    if (n > 0)
        return n * factorial(n - 1);
    else
        return 1;
}
\end{minted}
\end{frame}


\begin{frame}[t,fragile]\frametitle{Průběh výpočtu} 
Pokud zavoláme factorial(5), bude výpočet probíhat takto:

\begin{minted}[linenos=true]{c} 
factorial(5)
5 * (factorial(4))
5 * (4 * (factorial(3))))
5 * (4 * (3 * (factorial(2))))
5 * (4 * (3 * (2 * (factorial(1)))))
5 * (4 * (3 * (2 * (1 * (factorial(0))))))
5 * (4 * (3 * (2 * (1 * (1)))))
5 * (4 * (3 * (2 * (1))))
5 * (4 * (3 * (2)))
5 * (4 * (6))
5 * (24)
120
\end{minted}
\end{frame}


\begin{frame}[t,fragile]\frametitle{Najdi nulu v poli (cyklem)} 
\begin{minted}[linenos=true]{c} 
int najdi_nulu(int pole[], int delka) {
    int i;
    for (i = 0; i < delka; i++) {
        if (pole[i] == 0)
            return i;
    }
    return -1;
}

int main() {
    int cisla[] = {2,3,0,5,4};
    printf("%i\n", najdi_nulu(cisla, 5)); // 2
}
\end{minted}
\end{frame}



\begin{frame}[t,fragile]\frametitle{Najdi nulu v poli (rekurzí)} 
\begin{minted}[linenos=true]{c} 
int najdi_nulu(int pole[], int delka, int i) {
    if (i >= delka) { // nejsme uz za koncem pole?
        return -1; // hmm, tak jsme nulu nenasli.
    }
    else {
        if (pole[i] == 0) 
            return i; // vratime index prvni nuly
        else // jinak otestujeme prvek o bunku dale
            return najdi_nulu(pole, delka, i + 1);
    }
}
int main() {
    int cisla[] = {7,3,0,5,4};
    printf("%i\n", najdi_nulu(cisla, 5, 0)); // 2
}
\end{minted}
\end{frame}



\begin{frame}[t,fragile]{Úlohy}
\begin{center}
\vskip 1cm
{\Large Seznam úloh je na \url{http://KMIzp2.jdem.cz/}}
\vskip 2cm
\url{http://tux.inf.upol.cz/~havrlant/}
\end{center}
\end{frame}


\end{document}