\documentclass{beamer}
\usetheme[pageofpages=z,
          bullet=circle,
          titleline=true,
          alternativetitlepage=true,
          titlepagelogo=uplogo,
          ]{UVT}

\usepackage[utf8x]{inputenc}
\usepackage{czech}
\usepackage{minted}
% \usepackage{url}


\newenvironment{itemizex}%
  {\large \begin{itemize}%
    \setlength{\itemsep}{8pt}%
    \setlength{\parskip}{8pt}}%
  {\end{itemize}}

\newenvironment{enumeratex}%
  {\large \begin{enumerate}%
    \setlength{\itemsep}{6pt}%
    \setlength{\parskip}{6pt}}%
  {\end{enumerate}}

\newenvironment{itemizey}%
  {\large \begin{itemize}%
    \setlength{\itemsep}{6pt}%
    \setlength{\parskip}{6pt}}%
  {\end{itemize}}


\title{Základy programování 2 (KMI/ZP2 a KMI/UP2)}
\subtitle{Cvičení 8: Práce s binárními soubory}
\author{Lukáš Havrlant}
\date{9. dubna 2013}
\institute{Univerzita Palackého}

\begin{document}

\begin{frame}[t,plain]
\titlepage
\end{frame}


\begin{frame}[t,fragile]\frametitle{Textový vs. binární soubor} 
    \begin{itemizex}
        \item V binárních souborech jsou data zapsána ve stejné podobě, jako jsou v paměti.
        \item Číslo typu \texttt{int} 123456 v textovém souboru: 6 znaků, "123456"
        \item V binárním: 11110001001000000, velikost 4 bajty (pokud byl \texttt{int} 4bajtový)
        \item Při otevírání binárních souborů používáme navíc v módu \texttt{b}.
    \end{itemizex}
\end{frame}


\begin{frame}[t,fragile]\frametitle{Výhody a nevýhody binárních souborů} 
    \begin{itemizex}
        \item Binární soubory typicky zabírají méně místa než nekomprimovaný textový soubor.
        \item Práce s binárními soubory je rychlejší: např. nepotřebujeme znaky "123456" převádět na číslo.
        \item \dots ale ne vždy jsou přenositelné na jiný počítač kvůli různě velkým typům.
    \end{itemizex}
\end{frame}


\begin{frame}[t,fragile]\frametitle{Zápis binárních dat} 
    \begin{itemizey}
        \item Funkce pro zápis dat:
\begin{minted}[linenos=true]{c} 
size_t fwrite(void* odkud, size_t rozmer, 
                size_t pocet, FILE* soubor);
\end{minted}
        \item \texttt{odkud} je adresa začátku dat, která chceme zapsat. 
        \item \texttt{rozmer} je velikost jedné položky.
        \item \texttt{pocet} je počet položek, které chceme zapsat.
        \item \texttt{soubor} je ukazatel na soubor.
        \item Funkce vrací počet úspěšně zapsaných položek, tj. pokud vše proběhlo OK, vrací hodnotu \texttt{pocet}.
    \end{itemizey}
\end{frame}


\begin{frame}[t,fragile]\frametitle{Příklad zápisu} 
\begin{minted}[linenos=true]{c} 
int cisla[] = {-1, 1, 9, 15, 16, 31};

// otevreme soubor pro zapis binarnich dat
FILE* soubor = fopen("soubor.dat", "wb");

// zapiseme 6 polozek typu (velikosti) int,
// ktera zacinaji na adrese "cisla" (=zacatek pole)
fwrite(cisla, sizeof(int), 6, soubor);

fclose(soubor);
\end{minted}
\end{frame}


\begin{frame}[t,fragile]\frametitle{Obsah souboru} 
    \begin{itemizex}
        \item Binární soubor nemá smysl otevírat v běžném textovém editoru.
        \item Ale existují editory, které binární soubory zobrazí v hexa zápise:
        \item Po zápisu čísel -1, 1, 9, 15, 16, 31 bychom tak mohli dostat:
\begin{minted}[linenos=true]{bash} 
ffff ffff 0100 0000 0900 0000 0f00 0000
1000 0000 1f00 0000 
\end{minted}
    \end{itemizex}
\end{frame}


\begin{frame}[t,fragile]\frametitle{Příklad zápisu jedné proměnné} 
\begin{minted}[linenos=true]{c} 
double cislo = 3.14;
FILE* soubor = fopen("soubor.dat", "wb");

// Pozor, musime predat adresu. Identifikator pole je adresa
// ale u bezne promenne musime adresu vynutit operatorem &
// Pocet nastavime na 1, protoze ukladame 1 polozku.
fwrite(&cislo, sizeof(double), 1, soubor);

fclose(soubor);
\end{minted}
\end{frame}


\begin{frame}[t,fragile]\frametitle{Čtení binárních dat} 
    \begin{itemize}
        \item Funkce pro čtení dat: 
\begin{minted}[linenos=true]{c} 
size_t fread(void* kam, size_t rozmer, 
                size_t pocet, FILE* soubor);
\end{minted}
        \item \texttt{kam} je adresa paměti, kam se mají data načíst.
        \item \texttt{rozmer} udává velikost jedné přečtené položky.
        \item \texttt{pocet} je počet přečtených jednotek.
        \item \texttt{soubor} je ukazatel na soubor, ze kterého se čte.
        \item Funkce vrací počet přečtených jednotek -- hodnotu \texttt{pocet}, pokud vše proběhlo správně. Pokud vrátí jinou hodnotu, narazili jsme na konec souboru nebo nastala chyba.
    \end{itemize}
\end{frame}


\begin{frame}[t,fragile]\frametitle{Příklad čtení dat} 
\begin{minted}[linenos=true]{c} 
// Obsah souboru: -1, 1, 9, 15, 16, 31
int nactena_cisla[6];

// otevreme pro cteni v binarnim rezimu!
FILE* soubor = fopen("soubor.dat", "rb");

// Precteme 6 polozek typu int do pole nactena_cisla
// nactena_cislaje ukazatel na zacatek pole
fread(nactena_cisla, sizeof(int), 6, soubor);

// Vytiskne se: -1, 31
printf("%i, %i\n", cisla[0], cisla[5]);

fclose(soubor);
\end{minted}
\end{frame}


\begin{frame}[t,fragile]\frametitle{Příklad čtení jedné proměnné} 
\begin{minted}[linenos=true]{c} 
// V souboru je 3.14
// Otevreme soubor pro cteni v binarnim rezimu
FILE* soubor = fopen("soubor.dat", "rb");

// Promenne, do ktere nacteme cislo
double cislo;

// Musime predat adresu, kam to nacteme
// Nacteme jednu polozku o velikost double
fread(&cislo, sizeof(double), 1, soubor);

// Vypise se 3.14
printf("%g\n", cislo);
fclose(soubor);
\end{minted}
\end{frame}


\begin{frame}[t,fragile]\frametitle{Uložení/čtení více proměnných} 
\begin{minted}[linenos=true]{c} 
int cislo = 666, xcislo;
char retezec[] = "ahoj", xretezec[5];
// zapis do souboru
fwrite(&cislo, sizeof(int), 1, soubor); // zapiseme int
fwrite(retezec, sizeof(char), 5, soubor); // zapiseme retezec
fclose(soubor);
// soubor ma velikost 4 + 5 = 9 bajtu
// obsah: <int><char><char><char><char><char>
// zpetne precteme
fread(&xcislo, sizeof(int), 1, soubor); // precteme int
fread(xretezec, sizeof(char), 5, soubor); // precteme retezec
fclose(soubor);
printf("%i, %s\n", xcislo, xretezec);
\end{minted}
\end{frame}


\begin{frame}[t,fragile]\frametitle{Posun pozice v souboru} 
    \begin{itemizex}
        \item Pomocí funkce \texttt{fseek} můžeme hýbat s aktuálním pozicí v souboru.
\begin{minted}[linenos=true]{c} 
int fseek(FILE* soubor, long posun, int odkud);
\end{minted}
        \item \texttt{posun} počet bajtů (může být i záporné číslo) o kolik se má změnit aktuální pozice
        \item \texttt{odkud} udává výchozí pozici. Můžeme použít konstanty:
        \begin{itemize}
            \item SEEK\_SET -- od začátku souboru 
            \item SEEK\_CUR -- od aktuální pozice
            \item SEEK\_END -- od konce souboru
        \end{itemize}
    \end{itemizex}
\end{frame}


\begin{frame}[t,fragile]\frametitle{Příklad posunu pozice} 
\begin{minted}[linenos=true]{c} 
// Obsah souboru: -1, 1, 9, 15, 16, 31
FILE* soubor = fopen("soubor.dat", "rb");
int cislo;
// Posuneme se o dva inty dopredu od zacatku
fseek(soubor, 2 * sizeof(int), SEEK_SET);
fread(&cislo, sizeof(int), 1, soubor);
printf("%i\n", cislo); // 9

// Posuneme se o jeden int dorpedu od aktualni pozice
fseek(soubor, sizeof(int), SEEK_CUR);
fread(&cislo, sizeof(int), 1, soubor);
printf("%i\n", cislo); // 16
\end{minted}
\end{frame}


\begin{frame}[t,fragile]\frametitle{Posun zpět} 
\begin{minted}[linenos=true]{c} 
// Obsah souboru: -1, 1, 9, 15, 16, 31
FILE* soubor = fopen("soubor.dat", "rb");
int cislo;
// Vratime se zpet o jeden int
fseek(soubor, -sizeof(int), SEEK_END);

// precteme posledni int v souboru
fread(&cislo, sizeof(int), 1, soubor);

// vytiskne se 31
printf("%i\n", cislo);
\end{minted}
\end{frame}


\begin{frame}[t,fragile]\frametitle{Zjištění pozice} 
    \begin{itemize}
        \item Funkce \texttt{ftell} vrací aktuální pozici v souboru.
\begin{minted}[linenos=true]{c} 
long ftell(FILE* soubor);
\end{minted}
        \item Příklad:
\begin{minted}[linenos=true]{c} 
FILE* soubor = fopen("soubor.dat", "rb");
int cislo;
// vytiskne 0, jsme na zacatku
printf("%lu\n", ftell(soubor));
fread(&cislo, sizeof(int), 1, soubor);
// Vytiskne 4, posunuli jsme se o 1 int (pokud ma int 4 B)
printf("%lu\n", ftell(soubor));
fseek(soubor, -sizeof(int), SEEK_END);
// Vytiskne 20
printf("%lu\n", ftell(soubor));
fclose(soubor);
\end{minted}
    \end{itemize}
\end{frame}


\begin{frame}[t,fragile]\frametitle{Standardní proudy} 
    \begin{itemizex}
        \item Funkce \texttt{printf} tiskne na tzv. standardní výstup. To není nic jiného, než ukazatel na soubor, tj. \texttt{FILE*}, který je definovaný v \texttt{stdio.h}:
\begin{minted}[linenos=true]{c} 
FILE* stdin;
FILE* stdout;
FILE* stderr;
\end{minted}
        \item Tyto zápisy jsou tak totožné:
\begin{minted}[linenos=true]{c} 
printf("%i, %g\n", 10, 5.47);
fprintf(stdout, "%i, %g\n", 10, 5.47);
\end{minted}
    \end{itemizex}
\end{frame}



\begin{frame}[t,fragile]{Úlohy}
\begin{center}
\vskip 1cm
{\Large Seznam úloh je na \url{http://KMIzp2.jdem.cz/}}
\vskip 2cm
\url{http://tux.inf.upol.cz/~havrlant/}
\end{center}
\end{frame}


\end{document}