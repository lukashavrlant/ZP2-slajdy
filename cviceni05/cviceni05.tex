\documentclass{beamer}
\usetheme[pageofpages=z,
          bullet=circle,
          titleline=true,
          alternativetitlepage=true,
          titlepagelogo=uplogo,
          ]{UVT}

\usepackage[utf8x]{inputenc}
\usepackage{czech}
\usepackage{minted}
% \usepackage{url}


\newenvironment{itemizex}%
  {\large \begin{itemize}%
    \setlength{\itemsep}{8pt}%
    \setlength{\parskip}{8pt}}%
  {\end{itemize}}

\newenvironment{enumeratex}%
  {\large \begin{enumerate}%
    \setlength{\itemsep}{6pt}%
    \setlength{\parskip}{6pt}}%
  {\end{enumerate}}

\newenvironment{itemizey}%
  {\large \begin{itemize}%
    \setlength{\itemsep}{6pt}%
    \setlength{\parskip}{6pt}}%
  {\end{itemize}}


\title{Základy programování 2 (KMI/ZP2 a KMI/UP2)}
\subtitle{Cvičení 5: Práce s preprocesorem}
\author{Lukáš Havrlant}
\date{12. března 2013}
\institute{Univerzita Palackého}

\begin{document}

\begin{frame}[t,plain]
\titlepage
\end{frame}


\begin{frame}[t,fragile]\frametitle{K čemu slouží preprocesor} 
    \begin{itemizex}
        \item Zpracovává zdrojový text ještě před překladačem.
        \item Provádí v podstatě jednoduché search\&replace.
        \item Řádky určené preprocessoru začínají znakem \#, za kterým není mezera. Příklad:
\begin{minted}[linenos=true]{c} 
#define PI 3.1415
\end{minted}
        \item Definujeme konstantu \texttt{PI} -- všechny výskyty slova \texttt{PI} budou nahrazeny za \texttt{3.1415}.
        \item Konstanty obvykle zapisujeme velkými písmeny. 
    \end{itemizex}
\end{frame}


\begin{frame}[t,fragile]\frametitle{Složitější příklady} 
    \begin{itemizex}
        \item Můžeme definovat i složitější konstanty:
\begin{minted}[linenos=true]{c} 
#define MY_FOR for(i = 0; i < delka; i++)

int i, delka = 5, cisla[] = {2, 4, 6, 8, 10};
MY_FOR {
    printf("%i\n", cisla[i]);
}
\end{minted}
        \item \dots něco takového je ale lepší nepoužívat, protože není na první pohled jasné, co \texttt{MY\_FOR} je. 
    \end{itemizex}
\end{frame}


\begin{frame}[t,fragile]\frametitle{Přepsání konstanty} 
\begin{minted}[linenos=true]{c} 
#define PI 3.14
#define PI 3.14159265359
printf("%g\n", PI);
// warning: "PI" redefined
\end{minted}

Musíme nejprve starou konstantu zrušit pomocí \texttt{\#undef}:

\begin{minted}[linenos=true]{c} 
#define PI 3.14
#undef PI
#define PI 3.14159265359
printf("%g\n", PI);
\end{minted}
\end{frame}


\begin{frame}[t,fragile]\frametitle{Dlouhá konstanta} 
Pokud se nějaká konstanta nevleze na jeden řádek, lze rozdělit na více řádků pomocí znaku zpětného lomítka na konci řádku
\begin{minted}[linenos=true]{c} 
#define DLOUHA_KONSTANTA 12345678.12345\
67890123456
\end{minted}
\end{frame}


\begin{frame}[t,fragile]\frametitle{Makra s parametry} 
    \begin{itemize}
        \item Z paradigmat programování znáte makra -- C nabízí něco trochu podobného. 
        \item Konstantu cyklu bychom mohli parametrizovat takto:
\begin{minted}[linenos=true]{c} 
#define myfor(delka) for(i = 0; i < delka; i++)
int i, n = 4, cisla[] = {2, 4, 6, 8, 10};
myfor(n + 1) {
    printf("%i\n", cisla[i]);
}

// Makro se "prepise" na:
for(i = 0; i < n + 1; i++) {
    printf("%i\n", cisla[i]);
}
\end{minted}
    \end{itemize}
\end{frame}


\begin{frame}[t,fragile]\frametitle{Obecná syntaxe maker} 
\begin{itemizex}
    \item Obecná syntaxe maker bez parametrů (konstant):
\begin{minted}[linenos=true]{c} 
#defince jmeno hodnota
\end{minted}
    \item Obecná syntaxe maker s parametry:
    \begin{minted}[linenos=true]{c} 
#define jmeno(arg_1, ..., arg_N) telo_makra
    \end{minted}
    \item Za jménem makra, pokud má parametry, nesmí být mezera. 
\end{itemizex}
\end{frame}


\begin{frame}[t,fragile]\frametitle{Hodnota vs. text} 
Zatímco u funkcí se vždy předala až výsledná hodnota, u maker se předává pouze samotný text. 
\begin{minted}[linenos=true]{c} 
#define soucet(a, b) a + b
int vysledek = soucet(3, 7 + 8);
printf("%i\n", vysledek); // 18

// makro se rozepise na
int vysledek = 3 + 7 + 8;

// nerozepise se na 
int vysledek = 3 + 15; // NE!!!
\end{minted}
\end{frame}


\begin{frame}[t,fragile]\frametitle{Problémy s makry: priorita} 
Problém s prioritou ilustrujeme na makru pro součin:

\begin{minted}[linenos=true]{c} 
#define soucin(a, b) a * b
int vysledek = soucin(2 + 3, 5 + 5);
printf("%i\n", vysledek); // 22

// prepise se na
int vysledek = 2 + 3 * 5 + 5;

// reseni: zavorky kolem argumentu a kolem celeho makra:
#define soucin(a, b) ((a) * (b))
// prepsalo by se na
int vysledek = ((2 + 3) * (5 + 5));
\end{minted}
\end{frame}


\begin{frame}[t,fragile]\frametitle{Problémy s makry: dvojí vyhodnocení}
Pokud používáme v makru jednu proměnnou dvakrát, může se stát, že ji dvakrát vyhodnotíme:

\begin{minted}[linenos=true]{c} 
#define nadruhou(x) ((x) * (x))
int i = 1;
nadruhou(++i);

// Prepise se na 
((++i) * (++i));
\end{minted}
\end{frame}


\begin{frame}[t,fragile]\frametitle{Jak se řeší problém dvojího vyhodnocení} 
    \begin{itemizey}
        \item Problém dvojího vyhodnocení se řeší špatně. 
        \item Můžeme argument vyhodnotit jednou a uložit do pomocné proměnné.
        \item Narazíme na další problém: abychom mohli uložit výsledek do pomocné proměnné, potřebujeme znát typ původní proměnné:
\begin{minted}[linenos=true]{c} 
#define nadruhou(x) ({int temp = x; temp * temp; })
\end{minted}
        \item Nebude fungovat, pokud předáme \texttt{double}.
        \item Museli bychom typ předat jako argument. 
    \end{itemizey}
\end{frame}


\begin{frame}[t,fragile]\frametitle{Rozšíření GCC: příkaz \texttt{typeof}} 
    \begin{itemizex}
        \item Překladač \texttt{GCC} nabízí příkaz \texttt{typeof}. 
        \item Vrací typ dané proměnné. 
\begin{minted}[linenos=true]{c} 
int a;
typeof(a) b; // promenna b je typu int
\end{minted}
        \item Toho můžeme využít v našem makru:
\begin{minted}[linenos=true]{c} 
#define nadruhou(x) ({typeof(x) t = x; t * t; })
printf("%i\n", nadruhou(8)); // 64
printf("%g\n", nadruhou(8.5)); // 72.25
\end{minted}
    \end{itemizex}
\end{frame}


\begin{frame}[t,fragile]\frametitle{Jak skutečně vyřešit problém dvojího vyhodnocení} 
Použít funkci.
\end{frame}


\begin{frame}[t,fragile]\frametitle{Podmíněný překlad} 
Pomocí příkazů \texttt{\#if}, \texttt{\#else}, \texttt{\#elif} a \texttt{\#endif} můžeme řídit překlad. Za příkazem \texttt{\#if} musí být nějaký konstantní výraz:
\begin{minted}[linenos=true]{c} 
#define MAC_OS_X 1

#if MAC_OS_X
printf("Steve Jobs");
#else
syntakticka chyba
#endif
\end{minted}

Všimněte si, že v \texttt{else} části jsme dokonce udělali syntaktickou chybu -- překladač se k ní ale nedostal. 
\end{frame}


\begin{frame}[t,fragile]\frametitle{Podmíněný překlad řízený definicí makra} 
Často se používá kratší forma, která jen testuje, zda makro existuje:

\begin{minted}[linenos=true]{c} 
#define WINDOWS
#ifdef WINDOWS
   #define JMENO "C:\\Data\\input.txt"
#else
   #define JMENO "/data/input.txt"
#endif
\end{minted}

K dispozici je i obrácená varianta \texttt{\#ifndef}, která zjišťuje, zda konstanta není definovaná.
\end{frame}


\begin{frame}[t,fragile]\frametitle{Kdy použít makra: krátké funkce} 
    \begin{itemizex}
        \item Místo příliš jednoduchých funkcí, protože samotné porovnání se provede rychle, ale volání funkce je drahé.
        \item Nevýhody jsou v zásadě tři:
        \begin{enumerate}
            \item Nemůžeme používat rekurzi.
            \item Roste velikost programu. 
            \item Snadno makro napíšeme špatně, např. problém dvojího vyhodnocení.
        \end{enumerate}
        \item Můžeme využít i líného vyhodnocování:
        \begin{minted}[linenos=true]{c} 
#define IMPLIES(x, y) (!(x) || (y))
        \end{minted}
    \end{itemizex}
\end{frame}


\begin{frame}[t,fragile]\frametitle{Kdy použít makra: Složitější konstrukce} 
\begin{itemizex}
    \item V C++ se občas můžeme setkat s makrem \texttt{foreach}, které umožňuje jednoduše iterovat přes kolekce. Ukázka:
\begin{minted}[linenos=true]{c} 
BOOST_FOREACH(char c, "Hello, world!") {
    printf("%c\n", c)
}
\end{minted}
\end{itemizex}

\end{frame}



\begin{frame}[t,fragile]\frametitle{Jak psát složitější makra} 
\begin{itemizex}
    \item Problém s psaním složitějších maker je, že nemůžeme napsat něco takového:
\begin{minted}[linenos=true]{c} 
#define SWAP(x, y, T) \
T tmp = (x); (x) = (y); (y) = tmp;
int a = 5, b = 10;
if (1)
    SWAP(a, b, int);
else
    printf("\n");
\end{minted}
\end{itemizex}
\end{frame}

\begin{frame}[t,fragile]\frametitle{Opakování: operátor čárka} 
    \begin{itemizex}
        \item Příklad zápisu:
\begin{minted}[linenos=true]{c} 
vyraz_1, vyraz_2
\end{minted}
        \item Nejprve se vyhodnotí první výraz, \textbf{poté} druhý výraz. 
        \item Výsledkem vyhodnocení sekvenčního výrazu je vyhodnocení posledního výrazu.
\begin{minted}[linenos=true]{c} 
int cislo = (5 + 6, 40 + 2, 100 + 100);
printf("%i\n", cislo); // 200
\end{minted}
    \end{itemizex}
\end{frame}


\begin{frame}[t,fragile]\frametitle{Když nestačí operátor čárka} 
    \begin{itemizex}
        \item Pokud nestačí operátor čárka a potřebujete napsat ještě složitější makro, použijte trik s \texttt{do ... while(0)}:


\begin{minted}[linenos=true]{c} 
#define SWAP(x, y, T) do\
{ T tmp = (x); (x) = (y); (y) = tmp; } while(0)

int a = 5, b = 10;
SWAP(a, b, int);
printf("%i\n", a); // 10
\end{minted}
    \end{itemizex}
\end{frame}


\begin{frame}[t,fragile]\frametitle{Když nestačí operátor čárka a máte GCC} 
    \begin{itemizex}
        \item Překladač GCC nabízí ještě jednu možnost, jak napsat složitější výraz.
        \item Použijeme klasický blok \texttt{\{vyraz1; vyraz2; \}} a ještě ho navíc obalíme kulatými závorkami: \texttt{(blok)}.
        \begin{minted}[linenos=true]{c} 
#define nadruhou(x) ({typeof(x) t = x; t * t; })

#define SWAP(x, y, T) \
({ T tmp = (x); (x) = (y); (y) = tmp; })
        \end{minted}
    \end{itemizex}
\end{frame}


\begin{frame}[t,fragile]{Úlohy}
\begin{center}
\vskip 1cm
{\Large Seznam úloh je na \url{http://KMIzp2.jdem.cz/}}
\vskip 2cm
\url{http://tux.inf.upol.cz/~havrlant/}
\end{center}
\end{frame}


\end{document}