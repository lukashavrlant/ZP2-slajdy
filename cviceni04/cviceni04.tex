\documentclass{beamer}
\usetheme[pageofpages=z,
          bullet=circle,
          titleline=true,
          alternativetitlepage=true,
          titlepagelogo=uplogo,
          ]{UVT}

\usepackage[utf8x]{inputenc}
\usepackage{czech}
\usepackage{minted}
% \usepackage{url}


\newenvironment{itemizex}%
  {\large \begin{itemize}%
    \setlength{\itemsep}{8pt}%
    \setlength{\parskip}{8pt}}%
  {\end{itemize}}

\newenvironment{enumeratex}%
  {\large \begin{enumerate}%
    \setlength{\itemsep}{6pt}%
    \setlength{\parskip}{6pt}}%
  {\end{enumerate}}

\newenvironment{itemizey}%
  {\large \begin{itemize}%
    \setlength{\itemsep}{6pt}%
    \setlength{\parskip}{6pt}}%
  {\end{itemize}}


\title{Základy programování 2 (KMI/ZP2 a KMI/UP2)}
\subtitle{Cvičení 4: Ukazatele na funkce}
\author{Lukáš Havrlant}
\date{5. března 2013}
\institute{Univerzita Palackého}

\begin{document}

\begin{frame}[t,plain]
\titlepage
\end{frame}


\begin{frame}[t,fragile]\frametitle{Funkce vyššího řádu ve Scheme} 
    \begin{itemizex}
        \item Funkce vyššího řádu: funkce, která bere jako parametr funkci nebo vrací funkci.
        \item Typickými příklady jsou funkce \texttt{map} a \texttt{filter}:
\begin{minted}[linenos=true]{scheme} 
> (map (lambda (x) (* x x)) '(1 2 3 4 5 6))
(1 4 9 16 25 36)
> (filter (lambda (x) ( > x 3)) '(1 2 3 4 5 6))
(4 5 6)
\end{minted}
    \end{itemizex}
\end{frame}


\begin{frame}[t,fragile]\frametitle{Funkce vyššího řádu v C} 
    \begin{itemizex}
        \item Identifikátor funkce = ukazatel na funkci. 
        \item Můžeme tak předat ukazatel na funkci.
        \item Problém: C je staticky typovaný jazyk, potřebujeme v deklaraci uvádět návratový typ a typy parametrů. 
        \item Výsledkem je ne zrovna čitelná deklarace (pozor na závorky kolem identifikátoru):
\begin{minted}[linenos=true]{c} 
int (*funkce)(int, int); // ukazatel na funkci
int *funkce(int, int);   // deklarace funkce, ktera
                         // vraci ukazatel na int
\end{minted}
    \end{itemizex}
\end{frame}


\begin{frame}[t,fragile]\frametitle{Obecná deklarace} 
    \begin{itemizey}
        \item Obecná deklarace ukazatele na funkci má tvar:
\begin{minted}[linenos=true]{c} 
navratovy_typ (*nazev)(typ1, typ2, ..., typn);
\end{minted}
        \item \texttt{navratovy\_typ} je návratový typ funkce
        \item \texttt{nazev} je název proměnné, do které budeme funkci ukládat.
        \item \texttt{typ1} a další jsou typy parametrů funkce, pokud funkce nějaké parametry má. 
        \item \texttt{double (*fun)(char*, int)} by tak byla deklarace ukazatele na funkci, která vrací typ \texttt{double} a bere dva parametry: první typu \texttt{char*} a druhý typu \texttt{int}. \texttt{fun} je identifikátor. 
    \end{itemizey}
\end{frame}


\begin{frame}[t,fragile]\frametitle{Příklad} 
\begin{minted}[linenos=true]{c} 
int soucet(int x, int y) {
    return x + y;
}
int soucin(int x, int y) {
    return x * y;
}
int main() {
    // promenna "funkce" je ukazatelem na fci, ktera 
    // vraci int a bere dva parametry, oba typu int
    int (*funkce)(int, int) = soucet;
    printf("%i\n", funkce(10, 20));   // 30
    funkce = soucin;
    printf("%i\n", funkce(10, 20));   // 200
}
\end{minted}
\end{frame}


\begin{frame}[t,fragile]\frametitle{Předání funkce v parametru funkce} 
    \begin{itemizex}
        \item Stačí správně deklarovat ukazatel na funkci: 
\begin{minted}[linenos=true]{c} 
int male_pismeno(char znak) {
    return znak >= 'a' && znak <= 'z';
}
//pokud 0. znak splnuje podminku, pak se ret vypise
void vypis(int (*podminka)(char), const char* ret) {
    if (podminka(ret[0])) {
        printf("%s\n", ret);
    }
}
int main() {
    vypis(male_pismeno, "ahoj");    // vypise "ahoj"
    vypis(male_pismeno, "LuSteLa"); // nic nevypise
}
\end{minted}
    \end{itemizex}
\end{frame}


\begin{frame}[t,fragile]\frametitle{Použít \texttt{typedef}} 
    \begin{itemizey}
        \item Konstrukce \texttt{typedef} nám umožňuje vytvářet aliasy pro typy. Opakování:
\begin{minted}[linenos=true]{c} 
typedef int cislo; // vytvorime alias pro typ int
cislo x = 15;      // muzeme 
\end{minted}
        \item Můžeme tak vytvořit alias pro ukazatel na funkci daného typu:
\begin{minted}[linenos=true]{c} 
typedef int (*PODM)(char);

void vypis_pokud(PODM podminka, const char* ret) {
    if (podminka(ret[0])) {
        printf("%s\n", ret);
    }
}
\end{minted}
    \end{itemizey}
\end{frame}


\begin{frame}[t,fragile]\frametitle{Pole funkcí} 
    \begin{itemizex}
        \item Aby deklarace nebyly málo nepřehledné, můžeme ještě vytvářet pole funkcí. 
\begin{minted}[linenos=true]{c} 
int soucet(int x, int y) {
    return x + y;
}
int soucin(int x, int y) {
    return x * y;
}
int main() {
    // deklarujeme pole, ktere obsahuje dve funkce
    int (*pole_funkci[2])(int, int); 
    pole_funkci[0] = soucet;
    pole_funkci[1] = soucin;
}
\end{minted}
    \end{itemizex}
\end{frame}


\begin{frame}[t,fragile]\frametitle{Pole funkcí: volání funkce} 
    \begin{itemizex}
        \item Pokud máme pole funkcí \texttt{pole}, pak funkci na \texttt{i}-tém indexu voláme jednoduše jako \texttt{pole[i](...)}
\begin{minted}[linenos=true]{c} 
int main() {
    int (*funkce[2])(int, int);
    funkce[0] = soucet;
    funkce[1] = soucin;
    printf("%i\n", funkce[0](2, 8)); // 10
    printf("%i\n", funkce[1](2, 8)); // 16
}
\end{minted}
    \end{itemizex}
\end{frame}


\begin{frame}[t,fragile]{Úlohy}
\begin{center}
\vskip 1cm
{\Large Seznam úloh je na \url{http://KMIzp2.jdem.cz/}}
\vskip 2cm
\url{http://tux.inf.upol.cz/~havrlant/}
\end{center}
\end{frame}


\end{document}