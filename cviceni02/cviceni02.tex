\documentclass{beamer}
\usetheme[pageofpages=z,
          bullet=circle,
          titleline=true,
          alternativetitlepage=true,
          titlepagelogo=uplogo,
          ]{UVT}

\usepackage[utf8x]{inputenc}
\usepackage{czech}
\usepackage{minted}
% \usepackage{url}


\newenvironment{itemizex}%
  {\large \begin{itemize}%
    \setlength{\itemsep}{8pt}%
    \setlength{\parskip}{8pt}}%
  {\end{itemize}}

\newenvironment{enumeratex}%
  {\large \begin{enumerate}%
    \setlength{\itemsep}{6pt}%
    \setlength{\parskip}{6pt}}%
  {\end{enumerate}}


\title{Základy programování 2 (KMI/ZP2 a KMI/UP2)}
\subtitle{Cvičení 2: Vícerozměrná pole}
\author{Lukáš Havrlant}
\date{19. února 2013}
\institute{Univerzita Palackého}

\begin{document}

\begin{frame}[t,plain]
\titlepage
\end{frame}


\begin{frame}[t,fragile]\frametitle{Z minula: nezapomínejte na \texttt{return}} 
\begin{itemize}
    \item Ve Scheme jste nemuseli explicitně psát \texttt{return}, pokud jste rekurzivně volali nějakou funkci:
\begin{minted}[linenos=true]{scheme} 
(define factorial 
    (lambda (x)
        (if (= x 0) 
            1
            (* x (factorial (- x 1))))))
\end{minted}
    \item V Céčku příkaz \texttt{return} musíte uvádět!
\begin{minted}[linenos=true]{c} 
int factorial(int x) {
    if (x == 0) return 1;
    else return x * factorial(x - 1);
}
\end{minted}
    \item Pokud to v C funguje bez \texttt{return}u, tak je to náhoda.
\end{itemize}
\end{frame}


\begin{frame}[t,fragile]\frametitle{Jednorozměrné pole: opakování} 
  \begin{itemizex}
    \item Obecná definice:
    \begin{minted}[linenos=true]{c} 
typ nazev[pocet_prvku];
    \end{minted}
    \item Příklad:
    \begin{minted}[linenos=true]{c} 
int prvocisla[7];
double odmocniny[10];

int suda_cisla[5] = {2, 4, 6, 8, 10};
int licha_cisla[] = {1, 3, 5};
double dane[] = {16.5, 18, 21.75};
    \end{minted}
  \end{itemizex}
\end{frame}



\begin{frame}[t,fragile]\frametitle{Jednorozměrné pole: přístup k prvkům} 
\vskip 0.4cm
\begin{itemizex}
    \item Pomocí operátoru \texttt{[]}
    \item Pokud chceme získat/nastavit 4. prvek pole, použijeme kód:
    \begin{minted}[linenos=true]{c} 
int pole[] = {1, 2, 3, 4, 5, 6, 7};
int ctvrty_prvek = pole[4];
pole[2] = 0;
    \end{minted}
    \item Pole je číslováno od nuly, ne od jedničky!
    \item Obecně nelze zjistit délku pole. 
  \end{itemizex}
\end{frame}


\begin{frame}[t,fragile]\frametitle{Dvourozměrné pole} 
    \begin{itemizex}
        \item Slouží k uložení nějaké dvourozměrné tabulky, např.
$$
\begin{pmatrix}
2&5&6\\
3&1&9
\end{pmatrix}
$$
        \item Tabulka (matice) má 2 řádky a 3 sloupce, dvourozměrné pole deklarujeme takto:
\begin{minted}[linenos=true]{c} 
int matice[2][3];
int s_inicializaci[2][3] = {{2, 5, 6}, {3, 1, 9}};
\end{minted}
        \item První způsob alokoval místo pro šest prvků, ale nevynuloval ho. Druhý způsob alokoval místo pro šest prvků, které tam hned uložil. 
    \end{itemizex}
\end{frame}


\begin{frame}[t,fragile]\frametitle{Přístup k prvkům} 
    \vskip 1cm
    \begin{itemizex}
        \item Použijeme vícekrát operátor \texttt{[]}.
\begin{minted}[linenos=true]{c} 
int matice[2][3] = {{2, 5, 6}, 
                    {3, 1, 9}};
int a = matice[0][2]; // 6
int b = matice[1][0]; // 3
\end{minted}
        \item Přístup pomocí čárky nefunguje: \texttt{matice[1, 0]}. 
    \end{itemizex}
\end{frame}


\begin{frame}[t,fragile]\frametitle{Příklad} 
\begin{itemizex}
    \item Zbytečný text, který je tu jen proto, aby se následující kód zobrazil níže a zobrazil se tak celý na tabuli, protože promítací plátno je rozbité. 

\begin{minted}[linenos=true]{c} 
int matice[2][3], i, j;
for (i = 0; i < 2; i++)
    for (j = 0; j < 3; j++)
        matice[i][j] = (i + 1) * j;
\end{minted}
    \item Dostaneme takové dvourozměrné pole:
$$
\begin{pmatrix}
0&1&2\\
0&2&4
\end{pmatrix}
$$
\end{itemizex}
\end{frame}


\begin{frame}[t,fragile]\frametitle{Jak se pole ukládá v paměti} 
    \begin{itemize}
        \item Dvourozměrné pole lze simulovat jednorozměrným polem. Pole
$$
matice=\begin{pmatrix}
2&5&6\\
3&1&9
\end{pmatrix}
$$
        můžeme uložit pomocí jednorozměrného pole takto:

$$
vektor=\begin{pmatrix}
2&5&6&3&1&9
\end{pmatrix}
$$
        \item Takto se prvky dvourozměrného pole reálně uchovávají v paměti. 
        \item Pokud chceme získat prvek \texttt{matice[a][b]} u jednorozměrného pole, spočítáme jeho index jako $a\cdot S+b$, kde $S$ je počet sloupců. Takže \texttt{matice[1][2]} odpovídá prvku \texttt{vektor[1 * 3 + 2]}.
    \end{itemize}
\end{frame}



\begin{frame}[t,fragile]\frametitle{Vztah dvourozměrných polí a ukazatelů} 
    \begin{itemize}
        \item Definujme: \texttt{int pole[5]; int matice[3][7];}
        \item Pak \texttt{pole} je ukazatel na první prvek pole.
        \item \texttt{matice} je adresa dvourozměrného pole. 
        \item \texttt{matice[2]} je adresa třetího řádku / ukazatel na první prvek třetího řádku.
        \item \texttt{*(matice[2])} vrátí první prvek třetího řádku. 
    \end{itemize}

\begin{verbatim}
int pole[2][4] = {{1, 3, 5, 7}, {2, 4, 6, 8}};
Uloženo jako:
{1, 3, 5, 7, 2, 4, 6, 8}
 ^           ^
 pole[0]     pole[1]
 ^
 pole
\end{verbatim}
\end{frame}


\begin{frame}[t,fragile]\frametitle{Pole jako parametr funkce} 
    \vskip 1cm
    \begin{itemizex}
        \item Pokud chceme předat dvourozměrné pole jako parametr funkce, musíme vždy udat druhý rozměr:
\begin{minted}[linenos=true]{c} 
int funkce(int matice[][5]) { ... }
\end{minted}
        \item Proč? Když pak napíšeme \texttt{matice[3][0]}, tak program musí vědět, kolik sloupců má jeden řádek, aby se mohl posunout o správný počet buněk v \uv{jednorozměrném poli}.
    \end{itemizex}
\end{frame}


\begin{frame}[t,fragile]\frametitle{Příklad: vypsání pole} 
\begin{minted}[linenos=true]{c} 
// Vytiskne prvky dvourozmerneho pole o 4 sloupcich
void vypis_pole(int pole[][4], int radky) {
    int i, j;
    for (i = 0; i < radky; i++) {
        for (j = 0; j < 4; j++) {
            printf("%i, ", pole[i][j]);
        }
        printf("\n");
    }
}
// Priklad vystupu pro {{1, 3, 5, 7}, {2, 4, 6, 8}}:
// 1, 3, 5, 7, 
// 2, 4, 6, 8, 
\end{minted}
\end{frame}



\begin{frame}[t,fragile]\frametitle{Vícerozměrná pole} 
    \vskip 1cm
    \begin{itemizex}
        \item Můžeme samozřejmě vytvořit i troj/čtyř/n-rozměrné pole.
        \item Obecná syntaxe:
\begin{minted}[linenos=true]{c} 
typ identifikator[rozmer1][rozmer2]...[rozmer_n];
\end{minted}
        \item Pokud používáme vícerozměrné pole jako parametr, musíme specifikovat všechny rozměry kromě prvního:
\begin{minted}[linenos=true]{c} 
float funkce(char pole[][4][6][1002][42]) { ... }
\end{minted}
    \end{itemizex}
\end{frame}


\begin{frame}[t,fragile]{Úlohy}
\begin{center}
\vskip 1cm
{\Large Seznam úloh je na \url{http://KMIzp2.jdem.cz/}}
\vskip 2cm
\url{http://tux.inf.upol.cz/~havrlant/}
\end{center}
\end{frame}


\end{document}