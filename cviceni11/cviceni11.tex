\documentclass{beamer}
\usetheme[pageofpages=z,
          bullet=circle,
          titleline=true,
          alternativetitlepage=true,
          titlepagelogo=uplogo,
          ]{UVT}

\usepackage[utf8x]{inputenc}
\usepackage{czech}
\usepackage{minted}
% \usepackage{url}


\newenvironment{itemizex}%
  {\large \begin{itemize}%
    \setlength{\itemsep}{8pt}%
    \setlength{\parskip}{8pt}}%
  {\end{itemize}}

\newenvironment{enumeratex}%
  {\large \begin{enumerate}%
    \setlength{\itemsep}{6pt}%
    \setlength{\parskip}{6pt}}%
  {\end{enumerate}}

\newenvironment{itemizey}%
  {\large \begin{itemize}%
    \setlength{\itemsep}{6pt}%
    \setlength{\parskip}{6pt}}%
  {\end{itemize}}


\title{Základy programování 2 (KMI/ZP2 a KMI/UP2)}
\subtitle{Cvičení 11: Bitové operátory a bitová pole}
\author{Lukáš Havrlant}
\date{23. dubna 2013}
\institute{Univerzita Palackého}

\begin{document}

\begin{frame}[t,plain]
\titlepage
\end{frame}


\begin{frame}[t,fragile]\frametitle{Binární soustava} 
    \begin{itemizex}
        \item Číslo v desítkové soustavě můžeme zapsat v binární soustavě:
\begin{eqnarray*}
157_{10} &=& 10011101_2\\
20_{10} &=& 10100_2
\end{eqnarray*}
        \item Proč? Protože
$$
\underline{\textbf{0}}\cdot 2^0 + \underline{\textbf{0}}\cdot 2^1 + \underline{\textbf{1}}\cdot 2^2 + \underline{\textbf{0}}\cdot 2^3 + \underline{\textbf{1}}\cdot 2^4 = 4 + 16 = 20
$$
    \end{itemizex}
\end{frame}


\begin{frame}[t,fragile]\frametitle{Logické operátory} 
    \begin{itemizex}
        \item Z logiky známe logické spojky/operátory \uv{a zároveň} a \uv{nebo}. 
\begin{eqnarray*}
\mbox{nebo}(x, y) &=&  \max(x, y)\\
\mbox{a zároveň}(x, y) &=&  \min(x, y)\qquad (= x \cdot y)
\end{eqnarray*}
        \item Za $x$ a $y$ dosazujeme 1 (pravda) nebo 0 (nepravda).
        \item Tedy \texttt{nebo(1, 0) = 1}, ale \texttt{a\_zároveň(1, 0) = 0}.
    \end{itemizex}
\end{frame}

\begin{frame}[t,fragile]\frametitle{Bitový vektor} 
    \begin{itemizex}
        \item Logické operátory můžeme zobecnit tak, aby pracovaly s bitovým vektorem.
        \item Bitový vektor = posloupnost nul a jedniček.
\begin{verbatim}
          100011                       100011
a zároveň 100110                  nebo 100110
          ------                       ------
          100010                       100111
\end{verbatim}
    \end{itemizex}
\end{frame}


\begin{frame}[t,fragile]\frametitle{Číslo jako bitový vektor} 
    \begin{itemize}
        \item Číslo (integer) v C můžeme vidět jako bitový vektor.
\begin{eqnarray*}
20_{10} &=& 10100_2\\
17_{10} &=& 10001_2
\end{eqnarray*}
        \item Tedy můžeme zapsat \uv{\texttt{20 a zároveň 17}}. 
        \item V C k tomu slouží operátor \texttt{\&} (a zároveň) a \texttt{|} (nebo).
\begin{minted}[linenos=true]{c} 
int x = 20 & 17; // x = 16
\end{minted}
\begin{verbatim}
  10100
a 10001
-------
  10000
\end{verbatim}
        \item Přitom platí: $10000_2=16_{10}$.
    \end{itemize}
\end{frame}


\begin{frame}[t,fragile]\frametitle{Další bitové operátory} 
\begin{minted}[linenos=true]{c} 
x = 20 | 17; // x = 21
x = 20 ^ 17; // x = 5
// vsechny bitove operatory pracuji jen s celymi cisly
\end{minted}

\begin{verbatim}
     10100                        10100
nebo 10001                    XOR 10001
----------                        -----
     10101                        00101
\end{verbatim}
\begin{eqnarray*}
10101_2 &=& 21_{10}\\
00101_2 &=& 5_{10}
\end{eqnarray*}
    Stříška \texttt{\^} je bitový operátor nonekvivalence/XOR. Platí, že \texttt{xor(x,y) = 1} právě tehdy, když $x \ne y$ (\uv{ostré nebo}).
\end{frame}


\begin{frame}[t,fragile]\frametitle{Příklad: Chmod} 
    \begin{itemizex}
        \item Příkaz \texttt{chmod} slouží ke změně přístupových práv souboru. 
        \item Můžeme nastavit tři různé oprávnění: čtení, zápis, spuštění. 
        \item Realizováno jako bitový vektor o délce tři: čtení-zápis-spuštění. 
        \item Vektor \texttt{011} tak značí: nemůžeme číst, můžeme zapisovat a spouštět.
        \item Typicky nastavujeme pomocí celých čísel 0--7. $011_2=3_{10}$.
    \end{itemizex}
\end{frame}


\begin{frame}[t,fragile]\frametitle{Příklad: převod na velká/malá písmena} 
    \begin{itemizex}
        \item Platí že hodnota \uv{a} minus \uv{A} je rovna 32 atd. Přitom $32_{10}=100000_2$ $\rightarrow$ písmena se liší v jediném bitu.
\begin{verbatim}
A = 65 = 1000001          S = 83  = 1010011
a = 97 = 1100001          s = 115 = 1110011
\end{verbatim}
\begin{minted}[linenos=true]{c} 
printf("%c\n", 'A' | 32); // Vytiskne se: "a"
printf("%c\n", 'a' & 95); // Vytiskne se: "A"
\end{minted}
    \item V prvním případě jsme nastavili bit na 1, ve druhém na 0.
\begin{verbatim}
     1000001                  1100001
nebo 0100000                a 1011111
     -------                  -------
     1100001                  1000001
\end{verbatim}
    \end{itemizex}
\end{frame}


\begin{frame}[t,fragile]\frametitle{Příklad: množiny} 
    \begin{itemizey}
        \item Pomocí bitových operátorů lze reprezentovat množinu.
        \item Mějme množinu čísel od 0 do 7 (univerzum).
        \item Nějakou podmnožinu bychom mohli reprezentovat jako bitový vektor. 1 = obsahuje prvek na dané pozici, 0 = neobsahuje. 
$$
A=\{1,2,7\}, B = \{5,6,7\}, C=\{0,1,2,3,4,5,6,7\}, D = \emptyset
$$
\begin{eqnarray*}
A &=& 01100001\\
B &=& 00000111\\
C &=& 11111111\\
D &=& 00000000
\end{eqnarray*}
    \end{itemizey}
\end{frame}


\begin{frame}[t,fragile]\frametitle{Výhody bitové reprezentace} 
    \begin{itemizex}
        \item Můžeme implementovat klasické množinové operace pomocí bitových operátorů.
        \item A to pak bude sakramentsky rychlé.
        \item Např. průnik můžeme implementovat jen pomocí operátoru \texttt{\&}. 
$$
A \cap B = \{1,2,7\} \cap \{5,6,7\} = \{7\}
$$

\begin{verbatim}
  01100001
a 00000111
  --------
  00000001
\end{verbatim}
    \end{itemizex}
\end{frame}


\begin{frame}[t,fragile]\frametitle{Množina jako čísla} 
    \begin{itemizex}
        \item Jsme tak schopni reprezentovat množinu pomocí čísel:
\begin{eqnarray*}
A = 01100001_2 &=&  97_{10}\\
B = 00000111_2 &=& 7_{10}
\end{eqnarray*}
        \item V C tak můžeme napsat:
\begin{minted}[linenos=true]{c} 
char A = 97, B = 7;
char prunik = 97 & 7; // prunik = 1
\end{minted}
    \end{itemizex}
\end{frame}


\begin{frame}[t,fragile]\frametitle{Množina jako pole čísel} 
    \begin{itemizey}
        \item Pokud pro množinu použijeme typ \texttt{int}, jsme omezeni jeho velikostí.
        \item Pokud má \texttt{int} 4 bajty, můžeme ho použít pro uložení max 32prvkové množiny. 
        \item Řešení: použít pole intů/charů. 
\begin{minted}[linenos=true]{c} 
 unsigned char mnozina[] = {27, 61};
 \end{minted} 
        \item Protože $27_{10}=00011011_2, 61_{10}= 00111101_2$, tak tím získáme množinu \texttt{0001101100111101}.
        \item Průnik takové množiny bychom museli počítat po složkách: \texttt{mnA[0] \& mnB[0]; mnA[1] \& mnB[1];} atd. 
    \end{itemizey}
\end{frame}


\begin{frame}[t,fragile]\frametitle{Bitové posuny} 
    \begin{itemize}
        \item \texttt{>>} je bitový posun vpravo, \texttt{<<} je posun vlevo. 
        \item Příklad: \texttt{00001011 << 1} se bude rovnat \texttt{00010110}.
        \item Všechny bity posuneme a \uv{díry} zaplníme nulami. Takže \texttt{1111 >> 3 = 0001}
        \item Pro představu -- v desítkové soustavě by platilo: \texttt{0123 << 1 = 1230}, což je totéž jako $123 \cdot 10$.
        \item \texttt{<< 1} je totéž jako násobení dvěma. \texttt{<< x} je pak totéž jako násobení $2^x$. \texttt{>> x} je naopak dělení $2^x$.
\begin{minted}[linenos=true]{c} 
printf("%i\n", 15 << 1); // 30
printf("%i\n", 5 << 3); // 40
printf("%i\n", 32 >> 2); // 8
\end{minted}
    \end{itemize}
\end{frame}


\begin{frame}[t,fragile]\frametitle{Negace} 
    \begin{itemizex}
        \item Tilda \texttt{\~} je operátor negace. \texttt{\~\, 10001110 = 01110001}
        \begin{minted}[linenos=true]{c} 
printf("%i\n", ~0); // Vypise: -1
        \end{minted}
    \end{itemizex}
\end{frame}


\begin{frame}[t,fragile]\frametitle{Bitové pole} 
    \begin{itemizex}
        \item Bitové pole je podobné struktuře, akorát místo typů specifikujeme počet bitů. 
\begin{minted}[linenos=true]{c} 
typedef struct {
   unsigned den : 5;
   unsigned mesic : 4;
   unsigned rok : 7;
} DATUM;
\end{minted}
        \item Člen \texttt{den} má velikost 5 bitů, takže do něj můžeme uložit číslo v rozmezí $0$--$(2^5-1)$, tedy $0-31$.
    \end{itemizex}
\end{frame}


\begin{frame}[t,fragile]\frametitle{Bitové pole: pokračování} 
    \begin{itemizex}
        \item Celkový součet bitů nesmí přesáhnout hodnotu \texttt{sizeof(int)}.
        \item Položky můžou být buď celé znaménkové (\texttt{signed}), nebo celé neznaménkové (\texttt{unsigned}).
        \item Jinak se s tím pracuje stejně jako se strukturou:
\begin{minted}[linenos=true]{c} 
DATUM dnes = {23, 4, 2008 - 1980};
DATUM zitra = dnes;
zitra.den++;
dnes.mesic = 6;
dnes.rok = 2009 - 1980;
\end{minted}
    \end{itemizex}
\end{frame}


\begin{frame}[t,fragile]{Úlohy}
\begin{center}
\vskip 1cm
{\Large Seznam úloh je na \url{http://KMIzp2.jdem.cz/}}
\vskip 2cm
\url{http://tux.inf.upol.cz/~havrlant/}
\end{center}
\end{frame}


\end{document}