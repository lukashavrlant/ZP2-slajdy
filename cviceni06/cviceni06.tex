\documentclass{beamer}
\usetheme[pageofpages=z,
          bullet=circle,
          titleline=true,
          alternativetitlepage=true,
          titlepagelogo=uplogo,
          ]{UVT}

\usepackage[utf8x]{inputenc}
\usepackage{czech}
\usepackage{minted}
% \usepackage{url}


\newenvironment{itemizex}%
  {\large \begin{itemize}%
    \setlength{\itemsep}{8pt}%
    \setlength{\parskip}{8pt}}%
  {\end{itemize}}

\newenvironment{enumeratex}%
  {\large \begin{enumerate}%
    \setlength{\itemsep}{6pt}%
    \setlength{\parskip}{6pt}}%
  {\end{enumerate}}

\newenvironment{itemizey}%
  {\large \begin{itemize}%
    \setlength{\itemsep}{6pt}%
    \setlength{\parskip}{6pt}}%
  {\end{itemize}}


\title{Základy programování 2 (KMI/ZP2 a KMI/UP2)}
\subtitle{Cvičení 6: Funkce s proměnným počtem parametrů}
\author{Lukáš Havrlant}
\date{19. března 2013}
\institute{Univerzita Palackého}

\begin{document}

\begin{frame}[t,plain]
\titlepage
\end{frame}


\begin{frame}[t,fragile]\frametitle{Makro \texttt{assert}} 
    \begin{itemize}
        \item Typicky slouží k testování, zda funkce vrátila správný výsledek. Nachází se v knihovně \texttt{assert.h}
\begin{minted}[linenos=true]{c} 
void assert (int vyraz);
\end{minted}
        \item Pokud je \texttt{vyraz} nenulový, makro nic neudělá; pokud je \texttt{vyraz} roven nule, makro ukončí běh programu a vypíše chybu.
\begin{minted}[linenos=true]{c} 
int soucet(int a, int b) { return a + b; }
assert(2 == soucet(-3, 5)); // nic
assert(0 == soucet(-3, 3)); // nic
// Assertion failed: (1 == soucet(-3, 3))
assert(1 == soucet(-3, 3));
assert(1); // nic
assert(0); // Assertion failed: (0)
\end{minted}
    \end{itemize}
\end{frame}


\begin{frame}[t,fragile]\frametitle{Funkce \texttt{printf}} 
    \begin{itemizex}
        \item Zatím jsme definovali funkce, které měly přesný počet parametrů:
\begin{minted}[linenos=true]{c} 
int soucet(int a, int b) { return a + b; }
\end{minted}
        \item Funkce součet má dva parametry, oba typu \texttt{int}.
        \item Funkce \texttt{printf} se ale chová jinak:
\begin{minted}[linenos=true]{c} 
printf("%i, %g, %c", 45, 2.45, 'a'); // 4 parametry
printf("Ahoj Svete%c", '!'); // 2 parametry
\end{minted}
        \item Má různý počet parametrů. 
    \end{itemizex}
\end{frame}



\begin{frame}[t,fragile]\frametitle{Jak definovat FPPP} 
    \begin{itemizex}
        \item (FPPP = Funkce s proměnným počtem parametrů)
        \item K definici používáme tři tečky: \texttt{...}
        \item Funkce \texttt{printf} by mohla mít tuto deklaraci:
\begin{minted}[linenos=true]{c} 
int printf(const char* format, ...);
\end{minted}
        \item Parametr \texttt{format} představuje formátovací řetězec (tj. \texttt{"\%i, \%g, \%c"}), tři tečky značí, že dále můžeme použít libovolný počet parametrů. 
    \end{itemizex}
\end{frame}



\begin{frame}[t,fragile]\frametitle{Jak definovat FPPP podruhé} 
    \begin{itemizex}
        \item FPPP může mít i běžné \uv{pevné} parametry. 
        \item FPPP musí mít alespoň jeden \uv{pevný} parametr. 
        \item Deklarace funkce \texttt{suma} pro součet proměnlivého počtu čísel a následné volání by mohlo vypadat takto:
\begin{minted}[linenos=true]{c} 
int suma(int pocet, ...) { /* implementace */ }
int a = suma(4, 8, 9, 14, 57);
\end{minted}
        \item Toto nelze:
\begin{minted}[linenos=true]{c} 
int suma(...) { /* implementace */ } // Ne! 
\end{minted}
    \end{itemizex}
\end{frame}


\begin{frame}[t,fragile]\frametitle{Jak pracovat s FPPP} 
    \begin{itemizex}
        \item V knihovně \texttt{stdarg.h} jsou definována makra pro práci s parametry: \texttt{va\_list}, \texttt{va\_start}, \texttt{va\_arg}, \texttt{va\_end}.
        \item \texttt{va\_list} je datový typ pro uložení odkazu na parametry. 
        \item \texttt{va\_start} inicializuje proměnnou \texttt{va\_list}.
        \item \texttt{va\_arg} čte další parametr. 
        \item \texttt{va\_end} ukončuje práci s parametry. 
    \end{itemizex}
\end{frame}


\begin{frame}[t,fragile]\frametitle{Postup} 
    \begin{itemizex}
        \item Deklarujeme funkci \texttt{suma} s třemi tečkami a proměnnou typu \texttt{va\_list}.
        \item Inicializujeme proměnnou \texttt{parametry} pomocí makra \texttt{va\_start}, kde první parametr je proměnná \texttt{va\_list} a druhý parametr je poslední pevný parametr funkce \texttt{suma}.
    \end{itemizex}
\begin{minted}[linenos=true]{c} 
int suma(int pocet, ...) {
    int soucet = 0, i;
    va_list parametry;
    va_start(parametry, pocet);
}
\end{minted}
\end{frame}


\begin{frame}[t,fragile]\frametitle{Čtení parametrů} 
    \begin{itemize}
        \item Parametry čteme pomocí makra \texttt{va\_arg}. Ten má dva parametry: první je opět proměnná typu \texttt{va\_list} a druhým je typ, který má makro přečíst a vrátit.
        \item Makro \texttt{va\_arg} s každým zavoláním vrací další předaný parametr:
\begin{minted}[linenos=true]{c} 
int suma(int pocet, ...) {
    int soucet = 0, i;
    va_list parametry;
    va_start(parametry, pocet);
    printf("%i\n", va_arg(parametry, int)); // 10
    printf("%i\n", va_arg(parametry, int)); // 20
    printf("%i\n", pocet); // 3
}
suma(3, 10, 20, 30);
\end{minted}
    \end{itemize}
\end{frame}


\begin{frame}[t,fragile]\frametitle{Pozor na typy} 
My může FPPP zavolat s parametry libovolných typů. Funkce sama si to musí ošetřit.

\begin{minted}[linenos=true]{c} 
int suma(int pocet, ...) {
    int soucet = 0, i;
    va_list parametry;
    va_start(parametry, pocet);
    printf("%i\n", va_arg(parametry, int)); // 1356254312
    printf("%i\n", va_arg(parametry, int)); // 1402166392
    printf("%i\n", pocet); // 3
}
suma(3, 2.2, 3.3, 4);
\end{minted}
\end{frame}


\begin{frame}[t,fragile]\frametitle{Ukončení práce s parametry} 
Na konci práce s parametry zavoláme makro \texttt{va\_end}, které bere jeden parametr typu \texttt{va\_list}.

\begin{minted}[linenos=true]{c} 
int suma(int pocet, ...) {
    int soucet = 0, i;
    va_list parametry;
    va_start(parametry, pocet);
    /* secteni parametru */
    va_end(parametry);
    return soucet;
}
\end{minted}
\end{frame}


\begin{frame}[t,fragile]\frametitle{Jak poznáme, kolik parametrů jsme předali} 
    \begin{itemize}
        \item Nijak.
        \item Funkce musí mít v parametru počet předaných parametrů,
\begin{minted}[linenos=true]{c} 
for (i = 0; i < pocet; i++) {
    temp = va_arg(parametry, long double);
    /* zpracuj */
}
\end{minted}
        \item nebo musí být parametry něčím ukončeny (podobně jako je řetězec ukončen nulou).
\begin{minted}[linenos=true]{c} 
while(1) {
    temp = va_arg(parametry, int);
    if (temp != 0) {
        /* zpracuj */
    }
}
\end{minted}
    \end{itemize}
\end{frame}



\begin{frame}[t,fragile]{Úlohy}
\begin{center}
\vskip 1cm
{\Large Seznam úloh je na \url{http://KMIzp2.jdem.cz/}}
\vskip 2cm
\url{http://tux.inf.upol.cz/~havrlant/}
\end{center}
\end{frame}


\end{document}