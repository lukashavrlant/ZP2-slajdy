\documentclass{beamer}
\usetheme[pageofpages=z,
          bullet=circle,
          titleline=true,
          alternativetitlepage=true,
          titlepagelogo=uplogo,
          ]{UVT}

\usepackage[utf8x]{inputenc}
\usepackage{czech}
\usepackage{minted}
% \usepackage{url}


\newenvironment{itemizex}%
  {\large \begin{itemize}%
    \setlength{\itemsep}{8pt}%
    \setlength{\parskip}{8pt}}%
  {\end{itemize}}

\newenvironment{enumeratex}%
  {\large \begin{enumerate}%
    \setlength{\itemsep}{6pt}%
    \setlength{\parskip}{6pt}}%
  {\end{enumerate}}

\newenvironment{itemizey}%
  {\large \begin{itemize}%
    \setlength{\itemsep}{6pt}%
    \setlength{\parskip}{6pt}}%
  {\end{itemize}}


\title{Základy programování 2 (KMI/ZP2 a KMI/UP2)}
\subtitle{Cvičení 10: Piškvorky + dobré rady}
\author{Lukáš Havrlant}
\date{16. dubna 2013}
\institute{Univerzita Palackého}

\begin{document}

\begin{frame}[t,plain]
\titlepage
\end{frame}


\begin{frame}[t,fragile]\frametitle{Piškvorky} 
    \begin{itemizex}
        \item Na webu jsou ukázky. 
        \item Jsou tam i pomocné funkce ke stažení (nemusíte je použít.)
        \item Termín odevzdání: do konce zápočtového týdne.
        \item Můžete to odevzdávat jako jeden soubor, ale \dots bude lepší, když to bude ve více souborech.
    \end{itemizex}
\end{frame}



\begin{frame}[t,fragile]\frametitle{Hrací deska} 
    \begin{itemizex}
        \item Hrací deska má velikost $3\times3$
        \item Nejjednodušeji realizujete jako dvourozměrné dynamické pole charů.
        \item Používejte nějaké konstanty pro dva hráče + pro prázdné pole. Např. \texttt{\#define KRIZEK 'X'}
        \item Pokud si vzpomenete, můžete použít i \texttt{enum}
    \end{itemizex}
\end{frame}


\begin{frame}[t,fragile]\frametitle{Save \& load} 
    \begin{itemizex}
        \item Potřebujte funkci, která převede hrací desku na jeden řetězec. 
        \item Tento řetězec pak uložíte do textového souboru.
        \item Desku můžete textem reprezentovat pomocí devíti znaků: \texttt{XXO\_\_\_OXO}
        \item Obdobně naopak: potřebujete funkci, která vezme řetězec \texttt{XXO\_\_\_OXO}, vytvoří novou desku a překopíruje znaky na správná místa na desce.
    \end{itemizex}
\end{frame}


\begin{frame}[t,fragile]\frametitle{Drobnosti} 
    \begin{itemizex}
        \item Hra nesmí nikdy spadnout a nikdy se nesmí dostat do nekonzistentního stavu.
        \item Kontrolujte vstupy. Příkaz \texttt{dobry den} není platný příkaz, stejně tak \texttt{t8} není platná souřadnice. 
        \item Kontrolujte, zda můžete otevřít soubory pro zápis/čtení. 
    \end{itemizex}
\end{frame}


\begin{frame}[t,fragile]\frametitle{Rada: používejte vlastní typy} 
    \begin{itemizex}
        \item Typický způsob, jak reprezentovat políčko je jeden znak: \texttt{'X', 'O', '\_'}
        \item Nepoužívejte ale přímo typ \texttt{char}
        \item Vytvořte si alias: \texttt{typedef char Hrac} nebo \texttt{typedef char** HraciDeska}
        \item Pak můžete psát \texttt{Hrac na\_tahu = KRIZEK;}
        \item Nebo \texttt{Hrac dalsi\_na\_tahu(Hrac na\_tahu) \{...\}}
    \end{itemizex}
\end{frame}

\begin{frame}[t,fragile]\frametitle{Rada: rozdělejte program do funkcí} 
    \begin{itemizex}
        \item Nepokoušejte se vše nacpat do jedné jediné funkce, rozdělte program do více funkcí.
        \item Zhruba platí, že čím víc funkcí, tím lépe (nedovádět do extrémů\dots)
        \item Znovupoužitelnost (DRY -- don't repeat yourself)
        \item Lépe se v tom orientujete
        \item Nepotřebujte psát tolik komentářů
    \end{itemizex}
\end{frame}


\begin{frame}[t,fragile]\frametitle{Funkce má mít jeden účel} 
    \begin{itemizex}
        \item Jedna funkce má dělat jednu konkrétní věc. 
        \item A to je vše. 
        \item Jak by to nemělo vypadat:
\begin{minted}[linenos=true]{c} 
if (/* sileny vyraz kontrolujici souradnice */) {
    if (/* je pole obsazene? */) {
        // tady provedeneme tah
        hrac_na_tahu = /* prohodime hrac na tahu */
    }
}
// + osetreni chybovych stavu
\end{minted}
    \end{itemizex}
\end{frame}


\begin{frame}[t,fragile]\frametitle{Jak by to spíše mělo vypadat} 
\begin{minted}[linenos=true]{c} 
if (platne_souradnice(souradnice)) {
    if (!obsazeno(souradnice)) {
        tahni(souradnice, hraci_deska);
        hrac_na_tahu = dalsi_hrac(hrac_na_tahu);
    } 
}
// + osetreni chybovych stavu
\end{minted}
\end{frame}


\begin{frame}[t,fragile]\frametitle{Testujte!} 
    \begin{itemizex}
        \item V C máme \texttt{assert} nebo použijeme nějakou externí knihovnu.
        \item TDD = testy řízené programování -- extrémní verze testování: \textbf{nejdřív} píšeme testy a až \textbf{potom} píšeme kód.
        \item Méně extrémní verze: napíšeme kód (funkci) a připíšeme k ní testy.
        \item Testování pomáhá programátorovi během vývoje aplikace! 
    \end{itemizex}
\end{frame}


\begin{frame}[t,fragile]\frametitle{Výhody testování} 
    \begin{itemizex}
        \item Máte (relativní) jistotu, že když změníte kód v jedné funkci, že si nepomrvíte funkčnost jiné funkce.
        \item Můžete snadno refactorovat -- přepisovat kód do lepší podoby. 
        \item Klasický postup: napsat funkční kód, napsat testy, přepsat funkční kód, aby vypadal hezky. 
        \item Bez testů se po přepisu můžete jen modlit, že jste nic nezkazili. 
        \item (Jak můžete jít ve dvě ráno v klidu spát, když nevíte, zda jste poslední změnou v kódu nepokazili polovinu programu?!)
    \end{itemizex}
\end{frame}


\begin{frame}[t,fragile]\frametitle{Nevýhody testování} 
    \begin{itemizex}
        \item Je to zdlouhavé. 
        \item Psát testy je otravné.
        \item Pokud dojde k žádané změně programu, musí se přepsat související testy.
        \item Což je zdlouhavé.
        \item A otravné.
    \end{itemizex}
\end{frame}


\begin{frame}[t,fragile]\frametitle{Komentáře} 
    \begin{itemizex}
        \item Komentáře slouží k dokumentaci kódu.
        \item Nepoužíváme je nadbytečně, nemá smysl komentovat zřejmé věci:
\begin{minted}[linenos=true]{c} 
i++; // zvysime hodnotu i o +1
// alokujeme misto pro promennou slovo
slovo = (char*) malloc(sizeof(char) * 10);
while (znak = slovo[i++]) {} // projdeme cyklem slovo
slovo[i] = '\0'; // ukoncime slovo nulou
\end{minted}
    \end{itemizex}
\end{frame}


\begin{frame}[t,fragile]\frametitle{Co má smysl komentovat} 
    \begin{itemizex}
        \item K čemu slouží daná funkce. 
\begin{minted}[linenos=true]{c} 
// obrati slovo, "srp" -> "prs"
char* obrat(char* retezec) { ... }
\end{minted}
        \item Často existuje speciální syntaxe pro popis funkce, parametrů funkce a návratové hodnoty. 
        \item Speciální případy, které nejsou ze samotného kódu zřejmé.
        \item Často bývá lepší nepsat komentáře, ale mít v kódu správně pojmenované proměnné/funkce. 
    \end{itemizex}
\end{frame}

\begin{frame}[t,fragile]{Úlohy}
\begin{center}
\vskip 1cm
{\Large Seznam úloh je na \url{http://KMIzp2.jdem.cz/}}
\vskip 2cm
\url{http://tux.inf.upol.cz/~havrlant/}
\end{center}
\end{frame}


\end{document}